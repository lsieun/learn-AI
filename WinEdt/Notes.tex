\documentclass[UTF8]{ctexart}
\usepackage{amsmath}
   \author{刘森}
   \title{跟我学人工智能}
\begin{document}
    \maketitle
你好,世界 hello, world % This is comment

\section{机器学习算法}
\paragraph{1} 不同的算法,本身没有好坏之分,有的只是,根据不同的场景选择合适的算法。
\paragraph{2} 线性回归和Logistic回归,虽然听起来都叫作“回归”,但其实两者却是做不一样的事情:一个是做连续数据的预测,一个是做离散数据的预测;一个是真正做回归的,一个是做分类的,它们两个【用途】是完全不一样的。【如何推导出来?】线性回归是用高斯分布的方式推导出来,Logistic回归既然是做分类,就用Bnody分布,两点分布来推导出来。两者大的工具都是【最大似然估计】。在线性回归里面,要讨论一个东西:【最小二乘法的本质是什么】。或者说,为什么有最小二乘法呢?有没有最小三乘法呢?有没有最小四乘法呢?在【线 性回归】和【Logistic回归】中强调两个工具:【梯度下降算法】和【极大似然估计】。
    
\subsection{线性回归}
高斯分布 \\ 极大似然估计MLE \\ 最小二乘法的本质 \\

\subsubsection{什么是线性回归}
线性回归
$y=ax+b$\\

考虑多个变量情形,例如两个变量,$h_{\theta}(x)=\theta_{0}+\theta_{1}x_{1}+\theta_{2}x_{2}$ ,可以写成如下形式:

\begin{displaymath}
h_{\theta}(x)=\sum_{i=0}^{n}\theta_{i}x_{i}=\theta^{T}x
\end{displaymath}

其中,$\theta$展开后,呈现如下形式:

\[ \left( \begin{array}{c}
\theta_{0} \\
\theta_{1} \\
\theta_{2} \end{array} \right)\]

其中,$x$展开后,呈现如下形式:

%\[ \left( \begin{array}{c}
%\x_{0} \\
%\x_{1} \\
%\x_{2} \end{array} \right)\]
\[ \left( \begin{array}{c}
1 \\
x_{1} \\
x_{2} \end{array} \right)\]

上式中的$1$就表示$x_{0}$,而相应的$\theta_{0}$表示截距,是比较难以直接解释的。



再把上面的式子拿过来,

\begin{displaymath}
h_{\theta}(x)=\sum_{i=0}^{n}\theta_{i}x_{i}=\theta^{T}x
\end{displaymath}

在$h_{\theta}(x)$中,$x$看起来是【自变量】,但事实上是【样本】,所以$x$是已知的,而$\theta$是未知的,我们要通过某一种办法来求解出$\theta$,这个就是【线性回归要解决的问题】。

第05课《回归》00:10:30

目前讲的问题是【what】,即什么是线性回归。过一会儿,会讲【how】,用什么样的工具去求,如何去求的问题。

\subsubsection{使用极大似然估计解释最小二乘}

第05课《回归》00:20:00

使用极大似然估计解释最小二乘

\begin{displaymath}
y^{(i)}=\theta^{T}xY{(i)}+\epsilon^{(i)}
\end{displaymath}

the $\epsilon^{(i)}$ are distributed IID (independently and identically distributed) according to a Gaussian distribution (also called a Normal distribution) with mean $zero$ and some variance $\sigma^{2}$.

误差$\epsilon^{(i)}(1 \leq i \leq m)$是独立同分布的,服从均值为$0$,方差为某定值$\sigma^{2}$的【高斯分布】。原因:【中心极限定理】,可以查阅一下“中心极限定理的意义”。

似然函数 第05课《回归》00:21:21

首先,两边是相等的:

\begin{displaymath}
y^{(i)}=\theta^{T}x^{(i)}+\epsilon^{(i)}
\end{displaymath}

其中,$x^{(i)}$表示第$i$个【样本】,$\theta^{T}x^{(i)}$表示第$i$个样本的【预测值】,$y^{(i)}$表示第$i$个样本的【真实值】,而$\epsilon^{(i)}$表示第$i$个样本的误差。

根据【中心极限定理】,$\epsilon^{(i)}$应该是呈现一个高斯分布的形态。

\begin{displaymath}
p(\epsilon^{(i)})=\frac{1}{\sqrt{2\pi}\sigma}\exp(-\frac{(\epsilon^{(i)})^{2}}{2\sigma^{2}})
\end{displaymath}

另外,$\epsilon^{(i)} = y^{(i)} - \theta^{T}x^{(i)}$,此时将$\epsilon^{(i)}$代入上式:

\begin{displaymath}
p(y^{i}|x^{i};\theta)=\frac{1}{\sqrt{2\pi}\sigma}\exp(-\frac{(y^{(i)} - \theta^{T}x^{(i)})^{2}}{2\sigma^{2}})
\end{displaymath}

如此一来,上式当中就没有误差$\epsilon$了,因此只要指定了$x$和$\theta$,就可以认为是一个$y$的分布。换句话讲,$y$其实服从的是【均值是$\theta^{T}x$,方差是某一个$\sigma$的高斯分布(正态分布)】。

那么,用什么可以估计这个$\theta$呢?答:【最大似然估计】。

在上面的公式中,$i$只是表示第$i$个样本,假设一共有$m$个样本,那么,【$m$个样本的似然估计】就可以表示为:

\begin{displaymath}
L(\theta)=\prod_{i=1}^{m}p(y^{i}|x^{i};\theta) = \prod_{i=1}^{m}\frac{1}{\sqrt{2\pi}\sigma}\exp(-\frac{(y^{(i)} - \theta^{T}x^{(i)})^{2}}{2\sigma^{2}})
\end{displaymath}

如此一来,怎么求$\theta$呢?直接对【似然函数】取对数,然后再想办法。

高斯的对数似然与最小二乘


\begin{equation}
\begin{aligned}
l(\theta)&=\log L(\theta)\\
&=\log \prod_{i=1}^{m}\frac{1}{\sqrt{2\pi}\sigma}\exp(-\frac{(y^{(i)} - \theta^{T}x^{(i)})^{2}}{2\sigma^{2}})\\
&=\sum_{i=1}^{m} \log \frac{1}{\sqrt{2\pi}\sigma}\exp(-\frac{(y^{(i)} - \theta^{T}x^{(i)})^{2}}{2\sigma^{2}})\\
&=m\log \frac{1}{\sqrt{2\pi}\sigma} - \frac{1}{\sigma^{2}} \cdot \frac{1}{2} \sum_{i=1}^{m}(y^{(i)} - \theta^{T}x^{(i)})^{2}
\end{aligned}
\end{equation}

现在,其实是通过【最大似然估计】加上【高斯分布】来得到了【最小二乘法】目标函数。换句话说,这就是解释的“为什么会有最小二乘法”这个概念。

\begin{equation}
\begin{aligned}
J(\theta)=\frac{1}{2} \sum_{i=1}^{m}(h_{\theta}(x^{(i)}) - y^{(i)})^{2}
\end{aligned}
\end{equation}

\subsubsection{$\theta$的解析式的求解过程}

$\theta$的解析式的求解过程 第05课《回归》00:29:52

\subsection{逻辑回归:分类问题的首选算法}
    
\subsection{工具}
梯度下降算法\\
极大似然估计\\
        
    \subsection{Softmax}
    \section{Tensorflow}
    \subsection{安装Tensorflow}

        conda install tensorflow

        pip install tensorflow

    \section{数学知识}
    1.LATEX控制序列的概念(类似于函数) \\ 控制序列可以是作为命令:以“$\backslash$”开头,参数:必须参数和可选参数。



\end{document} 